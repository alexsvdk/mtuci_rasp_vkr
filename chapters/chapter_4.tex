

\newpage
\begin{center}
    \textbf{\large 4. FLUTTER ПРИЛОЖЕНИЕ}
\end{center}
\refstepcounter{chapter}
\addcontentsline{toc}{chapter}{4. FLUTTER ПРИЛОЖЕНИЕ}

Мобильное приложение для получения расписания МТУСИ создано на базе фреймворка Flutter 
и предназначено для студентов, преподавателей и сотрудников университета. 
Пользователи могут выбрать свою группу, аудиторию, преподавателя, предмет и даты, 
на которые они хотят получить расписание занятий и экзаменов. 
После выбора фильтров, приложение обращается к GraphQL API, чтобы получить данные о занятиях и экзаменах, 
соответствующих заданным фильтрам.

Приложение имеет удобный интерфейс, который позволяет просматривать расписание. 
Кроме того, приложение позволяет экспортировать расписание в формате ICS, и добавлять в календарь.
Для удобства использования приложение имеет функции быстрого поиска и фильтрации,
которые позволяют пользователям быстро находить нужную информацию.
Кроме того, приложение сохраняет последний запрос в кеше устройства,
чтобы пользователи могли быстро вернуться к предыдущим поискам.

Мобильное приложение для получения расписания МТУСИ является удобным и
эффективным инструментом для студентов, преподавателей и сотрудников университета,
которые хотят быть в курсе своего расписания занятий и экзаменов.

\section{Обзор технологий, использованных при разработке приложения}

\textbf{Flutter} - это фреймворк для разработки мобильных приложений,
созданный компанией Google.
Он позволяет создавать высокопроизводительные приложения для Android и iOS
с использованием одного и того же кода на языке программирования Dart.
Flutter использует собственный движок для отрисовки пользовательского интерфейса,
что позволяет достичь высокой производительности и кроссплатформенной совместимости.
Flutter также предоставляет широкий набор виджетов и инструментов для создания красивого
и интуитивно понятного пользовательского интерфейса.

\textbf{Реактивный подход} - это подход к разработке программного обеспечения,
который основан на обработке изменений данных. 
В реактивном подходе данные представляются в виде потоков,
которые изменяются во времени.
При изменении данных в потоке, происходит обновление пользовательского интерфейса.

\section{Процесс создания ICS файла}
\begin{enumerate}
\item Пользователь узнает URL ICS файла календаря через метод GraphQL \texttt{calendarByFilter}.
\item При первоначальном обращении по URL микросервис получает фильтры из URL и запрашивает у сервиса 
GraphQL API данные о занятиях и экзаменах, соответствующих заданным фильтрам. 
Затем микросервис генерирует ICS-файл на основе полученных данных и возвращает 
его пользователю и сохраняет в файловае распределенное хранилище S3.
\item При последующих обращениях по URL микросервис проверяет актуальность уже 
генерированного файла при помощи механизма ведения ревизий, и если файл все 
еще актуальный - возвращает HTTP код переадресации на ICS из файлового хранилища.
\end{enumerate}

\section{Результат разработки сервиса ICS API}
На данный момент сервис активен, и его можно использовать по ссылке \url{http://mtuci.stereobreeze.ru/ics/*.ics}, 
где \texttt{*} - это ID фильтра, по которому будет сгенерирован ICS файл.
Исходный код сервиса расположен на доступен по ссылке \url{https://github.com/alexsvdk/mtuci_rasp_back/tree/master/ics_backend}.


\textbf{Мобильное приложение} - это программа, которая разработана для работы на мобильных устройствах,
таких как смартфоны и планшеты. Они обычно загружаются из центра приложений на мобильном устройстве,
таком как App Store или Google Play, и устанавливаются на устройство.
Мобильные приложения могут выполнять различные задачи, включая игры, социальные сети,
продуктивные инструменты, торговые платформы, и т.д.
В том числе, мобильные приложения могут быть использованы для доступа участников учебного процесса к расписанию.

Мобильные приложения могут быть созданы для разных операционных систем, 
таких как iOS, Android, Windows Mobile и т.д.
Каждая операционная система имеет свои собственные инструменты и языки программирования для создания мобильных приложений.
Например, для создания мобильных приложений для iOS,
нужно использовать язык программирования Swift или Objective-C, а для Android - Java или Kotlin.

Мобильные приложения имеют множество преимуществ перед другими типами программ.
Они могут использовать различные функции мобильных устройств, такие как камера, GPS, датчики, уведомления и т.д.,
чтобы создать более интуитивный и персонализированный пользовательский опыт.
Также мобильные приложения могут работать в автономном режиме, без подключения к Интернету,
что делает их более удобными для использования в путешествиях или в местах, где связь ограничена.
