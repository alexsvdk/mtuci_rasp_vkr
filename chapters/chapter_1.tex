\newpage
\begin{center}
    \textbf{\large 1. ТЕОРЕТИЧЕСКИЕ АСПЕКТЫ СОЗДАНИЯ АВТОМАТИЗИРОВАННЫХ ИНФОРМАЦИОННЫХ СИСТЕМ}
\end{center}
\refstepcounter{chapter}
\addcontentsline{toc}{chapter}{1. ТЕОРЕТИЧЕСКИЕ АСПЕКТЫ СОЗДАНИЯ АВТОМАТИЗИРОВАННЫХ ИНФОРМАЦИОННЫХ СИСТЕМ}



\section{Компоненты автоматизированных информационных систем}
Автоматизированные информационные системы (АИС) обычно состоят из различных компонентов, 
включая клиентскую и серверную части \cite{Ais2020}.
Обычнор в состав АИС входят слеующие компоненты:

\begin{enumerate}
    \item \textbf{Базы данных} -- Это программы, используемые для хранения и организации данных, необходимых для работы АИС. 
    Они обеспечивают постоянное хранение информации и предоставляют средства для манипулирования данными, 
    такие как добавление, изменение и удаление записей.
    
    \item \textbf{Серверы} -- это прошграммы, которые обеспечивают обработку и хранение данных, 
    выполнение бизнес-логики и предоставление функциональности АИС. 
    Серверы обычно обрабатывают запросы от клиентских приложений и возвращают данные или результаты операций.
    
    \item \textbf{Клиентские приложения} -- это программы, которые запускаются на устройствах пользователей и предоставляют им доступ к функциональности АИС. 
    Клиентские приложения обычно предоставляют пользовательский интерфейс для взаимодействия с системой, 
    отправки запросов и получения данных.
    
    \item \textbf{Коммуникационные протоколы} -- Правила для передачи данных между клиентскими приложениями и серверами, такие как HTTP, TCP/IP и другие. 
    Эти протоколы определяют формат обмена данных и правила взаимодействия между клиентом и сервером.
\end{enumerate}

Выбор конкретных технологий при реализации компонентов системы имеет непосредственное влиянме на различные аспекты системы. Об этом будет рассказано в следующих разделах.
Прежде всего, важно провести исследование и обзор современных технологий, которые могут быть использованы при разработке АИС.



\section{Современные технологии баз данных}
Современные технологии баз данных предлагают разнообразные подходы и решения для хранения, организации и управления данными. 
Каждая технология имеет свои особенности и оптимально подходит для определенных типов задач.

\begin{enumerate}
    \item \textbf{Реляционные базы данных (RDBMS):} Реляционные базы данных, 
    такие как MySQL, PostgreSQL, Oracle, Microsoft SQL Server, 
    предоставляют структурированное хранение данных в виде таблиц с возможностью связывания и выполнения сложных запросов. 
    Они хорошо подходят для традиционных бизнес-приложений, где требуется точность, целостность и согласованность данных.
    \item \textbf{NoSQL базы данных:} NoSQL (Not Only SQL) базы данных, 
    такие как MongoDB, Cassandra, Redis, 
    предназначены для работы с большими объемами неструктурированных данных и обладают гибкой схемой хранения. 
    Они часто используются для реализации распределенных систем, обработки больших данных и кеширования.
    \item \textbf{Колоночные базы данных:} Колоночные базы данных, например Apache HBase или Amazon Redshift, 
    оптимизированы для хранения и обработки больших объемов данных по столбцам, что делает их идеальными для аналитических задач, OLAP (Online Analytical Processing) и хранилищ данных.
    \item \textbf{Графовые базы данных:} Графовые базы данных, такие как Neo4j или Amazon Neptune, 
    предназначены для хранения и обработки данных, организованных в виде графовых структур. 
    Они широко используются в задачах, где требуется анализ связей и отношений между данными, 
    таких как социальные сети, рекомендательные системы и семантический анализ.
    \item \textbf{In-Memory базы данных:} In-Memory базы данных, например Redis или Memcached, 
    хранят данные непосредственно в оперативной памяти компьютера, что обеспечивает быстрый доступ к данным. 
    Они широко используются для кеширования данных, обработки в реальном времени и ускорения операций чтения и записи.
\end{enumerate}

Выбор оптимальной базы данных зависит от множества факторов, 
таких как объем данных, требования к производительности, типы запросов и модель данных. 
Часто в комплексных системах применяется комбинация разных баз данных, 
чтобы достичь оптимального баланса между различными требованиями.

\section{Современные технологии для разработки серверной части}
Существует множество современных технологий для разработки серверной части приложений:

\begin{enumerate}
    \item 1. \textbf{Java с фреймворком Spring:} Java остается одним из наиболее популярных языков программирования для разработки серверных приложений. 
    Фреймворк Spring предоставляет широкий спектр инструментов и функциональности для создания масштабируемых и надежных приложений. 

    2. \textbf{Kotlin с фреймворком Ktor:} Фреймворк Ktor позволяет создавать эффективные и масштабируемые веб-серверы с использованием Kotlin. 
    Он предлагает лаконичный синтаксис, асинхронную обработку запросов и хорошую поддержку микросервисной архитектуры.
    
    3. \textbf{Node.js с фреймворком Express:} Node.js - это платформа, основанная на JavaScript, которая позволяет выполнять серверный JavaScript. 
    Она обеспечивает высокую производительность и эффективность благодаря однопоточной, неблокирующей модели выполнения. 
    Фреймворк Express является популярным выбором для разработки веб-приложений на основе Node.js.
    
    4. \textbf{Go с фреймворком Gin:} Go - это язык программирования, разработанный компанией Google, 
    который обеспечивает высокую производительность и эффективность. 
    Фреймворк Gin является легковесным и быстрым решением для разработки веб-приложений на основе Go. 
    
    5. \textbf{Python с фреймворком Django:} Python - это высокоуровневый язык программирования с чистым синтаксисом и мощными возможностями. 
    Фреймворк Django является одним из наиболее популярных фреймворков для разработки веб-приложений на Python. 
    Он предоставляет множество инструментов и функций для удобной разработки серверной части приложений, 
    включая ORM (объектно-реляционное отображение), авторизацию, маршрутизацию и административный интерфейс.
\end{enumerate}

Это лишь некоторые из множества современных технологий для разработки серверной части приложений.
Каждая из них имеет свои особенности и преимущества. 
Выбор зависит от требований проекта, опыта разработчиков и предпочтений команды.

Кроме выбора технологии, важно выбрать подходящую архитектуру для разработки серверной части приложения.
Монолитная архитектура и микросервисная архитектура - это два разных подхода к организации приложений.

\textbf{Монолитная архитектура} -- это традиционный подход к разработке приложений. 
Приложение разрабатывается и развертывается как единое целое. 
Все компоненты и функциональности приложения находятся внутри одной кодовой базы и запускаются как одно приложение на одном сервере или группе серверов. 
Типичный монолит состоит из модулей или компонентов, которые связаны друг с другом и работают в тесной связи.

Преимущества монолитной архитектуры включают простоту разработки и развертывания, 
так как все компоненты находятся в одном месте. 
Тестирование и масштабирование монолита также может быть более простым, 
поскольку приложение работает как единое целое. Однако, монолиты могут страдать от недостатков, 
таких как сложность поддержки, сложность масштабирования в отдельных компонентах и 
высокая связность между модулями, что может затруднять разработку в больших командах.

\textbf{Микросервисная архитектура} -- это относительно новый подход к разработке приложений. 
Приложение разбивается на небольшие, независимые и связанные сервисы. 
Каждый сервис является отдельным компонентом, который выполняет конкретную функцию или предоставляет определенный сервис. 
Сервисы могут разрабатываться и развертываться независимо друг от друга, 
а коммуникация между ними осуществляется посредством сетевых вызовов, 
часто используя HTTP/REST API или сообщения.
"В архитектуре микросервисов приложение строится как независимые компоненты" \cite{Kravchenko2022}.

Преимущества микросервисной архитектуры включают легкость масштабирования отдельных сервисов,
возможность независимого развертывания и масштабирования, более гибкую разработку и поддержку,
а также лучшую изоляцию и надежность. Однако, микросервисы также могут вносить сложности в управление,
тестирование, отладку и требуют хорошей организации коммуникации и согласованности между сервисами.

Выбор между монолитной и микросервисной архитектурой зависит от
конкретных требований проекта, его масштаба,
команды разработчиков и других факторов.



\section{Современные технологии для разработки клиентских приложений}
Существует множество современных технологий для разработки клиентских приложений, включая следующие:

\begin{enumerate}
    \item \textbf{JavaScript} -- является одним из наиболее популярных языков программирования для разработки клиентских приложений. 
    Современные фреймворки, такие как React, Angular и Vue.js, 
    позволяют создавать мощные и интерактивные веб-приложения. 
    Они предоставляют компонентный подход, управление состоянием и маршрутизацию, 
    а также поддерживают разработку одностраничных приложений (SPA).
    \item \textbf{React Native} -- это фреймворк, основанный на React, 
    который позволяет разрабатывать нативные мобильные приложения 
    с использованием JavaScript. Он позволяет повторно использовать код и создавать приложения для iOS и Android, 
    обеспечивая высокую производительность и нативный пользовательский интерфейс.
    \item \textbf{Flutter} -- это фреймворк, разработанный компанией Google, для создания кросс-платформенных мобильных и веб-приложений. 
    Он использует язык программирования Dart и предлагает свой собственный набор компонентов 
    и инструментов для создания красивых и высокопроизводительных пользовательских интерфейсов.
    \item \textbf{Xamarin} -- это фреймворк, позволяющий разрабатывать кросс-платформенные мобильные приложения 
    с использованием языка программирования C\# и платформы .NET. 
    Он предоставляет доступ к нативным API для различных платформ, таких как iOS, Android и Windows, 
    и обеспечивает максимальную переносимость кода между платформами.
    \item \textbf{Electron} -- это фреймворк для создания настольных приложений с использованием веб-технологий, таких как HTML, 
    CSS и JavaScript. Он позволяет разработчикам использовать технологии веб-разработки 
    для создания кросс-платформенных приложений для Windows, macOS и Linux.
\end{enumerate}

Типы клиентских приложений, которые можно создать с помощью этих технологий, 
включают веб-приложения, мобильные приложения для iOS и Android, настольные приложения для различных операционных систем, 
а также гибридные приложения, которые сочетают в себе возможности веб-приложений и 
нативных приложений. Выбор технологии зависит от требований проекта и
целевой платформы.



\section{Современные коммуникационные протоколы}
Существуют различные коммуникационные протоколы, которые широко используются в современной разработке приложений:

\begin{enumerate}
    \item \textbf{HTTP (Hypertext Transfer Protocol} -- является протоколом передачи данных, который используется для обмена информацией между клиентом и сервером в Интернете. 
    Он определяет стандартные методы запроса (GET, POST, PUT, DELETE и др.), а также структуру сообщений и правила взаимодействия. 
    HTTP обычно используется для передачи гипертекстовых документов, таких как веб-страницы, 
    но также может быть использован для обмена данных в формате JSON или XML.

    \item \textbf{API (Application Programming Interface)} -- представляет собой интерфейс, который определяет набор правил и протоколов 
    для взаимодействия между различными программными компонентами. 
    API позволяет приложениям обмениваться данными и вызывать функции друг друга для выполнения определенных задач. 
    Часто API используется для доступа к функциональности серверных приложений или веб-служб.
    
    \item \textbf{XML (eXtensible Markup Language)} -- является языком разметки, который используется для представления структурированных данных. 
    Он предоставляет гибкий формат для обмена информацией между системами и приложениями. 
    XML документы имеют иерархическую структуру, определяемую с помощью тегов и атрибутов. 
    XML часто используется в коммуникации между различными системами или для хранения данных в структурированном виде.
    
    \item \textbf{JSON (JavaScript Object Notation)} -- является форматом обмена данными, основанным на синтаксисе JavaScript. 
    Он представляет данные в виде пар ключ-значение и используется для передачи структурированных данных между клиентом и сервером. 
    JSON обычно является более легким и компактным, чем XML, и часто используется в веб-разработке и API.
    
    \item \textbf{GraphQL} -- это язык запросов и среда выполнения, разработанные Facebook (организация запрещена в РФ)
    \footnote{Facebook принадлежит компании Meta, признанной экстремистской организацией и запрещенной в РФ}. 
    Он позволяет клиентам запросить только нужные данные и предоставляет гибкость в определении структуры и содержания ответа. 
    GraphQL работает поверх HTTP и позволяет клиентам указывать, какие данные им требуются, 
    уменьшая объем сетевого трафика и повышая эффективность коммуникации.
    
\end{enumerate}

Выбор коммуникационного протокола или формата данных зависит от конкретных трeбований проекта.



\section{Ключевые критерии выбора технологий компонентов системы}
При выборе технологий для разработки системы или приложения следует учитывать несколько ключевых критериев:

\begin{enumerate}
    \item \textbf{Функциональность} -- определяет спектр возможностей и задач, которые система может выполнять. 
    Функциональность может варьироваться в зависимости от конкретных требований и целей системы,
    и выбор технологий может повлиять на доступные функции и способы их реализации.
    \item \textbf{Производительность} -- Способность АИС обрабатывать и отвечать на запросы и операции с высокой скоростью и эффективностью.
    Это может включать скорость обработки данных, время отклика, 
    пропускную способность и эффективное использование ресурсов системы, 
    таких как процессорное время, память и сетевая пропускная способность. 
    Выбор оптимальных технологий и алгоритмов может помочь достичь высокой производительности АИС.
    \item \textbf{Масштабируемость} -- Способность системы адаптироваться и масштабироваться с ростом нагрузки, объема данных и пользовательской базы. 
    Это может включать горизонтальное масштабирование (добавление дополнительных серверов или узлов) или вертикальное масштабирование (увеличение ресурсов на существующих серверах). 
    Выбор технологий, которые обеспечивают хорошую масштабируемость, позволяет системе эффективно работать в условиях растущих требований.
    \item \textbf{Удобство разработки} относится к уровню сложности и доступности инструментов, 
    фреймворков, документации и ресурсов, необходимых для разработки АИС. 
    Это включает удобство использования инструментов разработки, простоту интеграции с другими системами, наличие готовых компонентов или библиотек, 
    а также поддержку и сообщество разработчиков. Выбор технологий, которые обеспечивают удобство разработки, 
    может значительно повысить производительность и эффективность разработчиков, сократить время разработки и облегчить поддержку системы.
    \item \textbf{Совместимость и интеграция}. Важно, чтобы выбранная технология была совместима с другими используемыми в проекте компонентами или системами. 
    \item \textbf{Сообщество и экосистема}. Размер и активность сообщества разработчиков вокруг выбранной технологии. 
    Большое сообщество может предоставить поддержку, ресурсы, библиотеки и решения для типичных проблем разработки, что ускорит разработку и снизит риски.
    \item \textbf{Безопасность}. Выбранная технология должна предоставлять соответствующие проекту механизмы защиты данных, аутентификации и авторизации.
\end{enumerate}

Окончательный выбор технологий для компонентов АИС должен основываться на комплексном анализе требований проекта, 
учете доступных ресурсов и опыта команды разработчиков. Грамотный выбор технологий позволит создать эффективную, 
масштабируемую и удобную в использовании автоматизированную информационную систему.
