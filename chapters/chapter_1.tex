\newpage
\begin{center}
    \textbf{\large 1. СЕРВИС GRAPHQL API}
\end{center}
\refstepcounter{chapter}
\addcontentsline{toc}{chapter}{1. СЕРВИС GRAPHQL API}

Микросервис GraphQL API предназначен для обеспечения доступа мобильного приложения и других сервисов
к данным расписания, хранящимся в базе данных.

\section{Обзор технологий, использованных при разработке GraphQL API}

\textbf{Базы данных (БД)} - это средства хранения и организации данных,
обеспечивающие эффективный доступ к ним. Существует множество различных систем управления базами данных (СУБД),
таких как MySQL, PostgreSQL, Oracle, MS SQL Server и др.
Каждая СУБД имеет свои преимущества и недостатки, и выбор конкретной СУБД зависит от требований к системе.

\textbf{Доменное имя} - это уникальное имя, используемое для идентификации компьютера, сервера или другого устройства, подключенного к Интернету.
Оно представляет собой читаемое для людей название, которое соответствует числовому IP-адресу,
который используется компьютерами для идентификации друг друга в сети.

\textbf{Клиент} - это устройство или приложение, которое запрашивает информацию у другого устройства, называемого сервером. 
Клиент может быть любым устройством, например, компьютером, мобильным устройством или планшетом, 
которое подключено к сети Интернет или локальной сети. 
Клиентское устройство обычно используется для ввода данных или запросов, 
которые передаются на сервер для обработки.

\textbf{Сервер} - это компьютер или другое устройство, которое предоставляет информацию клиентским устройствам. 
Сервер может выполнять различные задачи, такие как обработка запросов клиентов, 
хранение данных, предоставление доступа к сети Интернет, управление ресурсами сети и многое другое.
Сервер может обслуживать множество клиентов одновременно, предоставляя им доступ к различным ресурсам. 
В общем случае, клиент и сервер образуют взаимодействующие компоненты,
которые работают совместно для выполнения различных задач в сети.
Клиент отправляет запросы на сервер, а сервер обрабатывает эти запросы и отвечает клиенту.
Коммуникация между клиентом и сервером может происходить по различным протоколам, например, HTTP, FTP, SMTP и т.д.

\textbf{API (Application Programming Interface)} - это набор правил, протоколов и инструментов, 
которые используются для создания программных приложений и позволяют им взаимодействовать с другими приложениями и системами. 
API предоставляет возможность разработчикам использовать функциональность и данные, 
предоставляемые другими приложениями или системами, 
без необходимости понимания внутренней работы их компонентов.

\textbf{GraphQL} - это язык запросов, разработанный Facebook для работы с API. 
Он позволяет клиентам запрашивать только те данные, которые им необходимы, и получать их в одном запросе. 
В отличие от REST API, где клиенты получают предопределенный набор данных, 
GraphQL предоставляет клиентам возможность выбирать конкретные поля и связи, 
которые им нужны, и получать только эти данные.
GraphQL имеет множество преимуществ по сравнению с REST API. 
Он позволяет более гибко и эффективно работать с данными, 
улучшает производительность и упрощает поддержку кода. Кроме того, 
GraphQL позволяет создавать более сложные запросы и связи между данными, 
что упрощает создание более мощных и гибких API.

\section{Результат разработки сервиса GraphQL API}

Сервис GraphQL API написан на языке программирования Kotlin с использованием фреймворка Spring и базы данных Mongo DB.
Сервис спроектирован по паттерну “Чистая архитектура”.
Работа с БД отделена логики обработки запросов при помощи паттерн проектирования “Репозиторий”.  

На данный момент в API существуют следующие публичные методы:
\begin{enumerate}
\item \texttt{findGroups} позволяет находить группы по заданным параметрам (например, по названию или описанию). Это может быть полезно для пользователей, которые ищут конкретную группу или группы, соответствующие определенным критериям.
\item \texttt{groupById} позволяет получать информацию о группе по ее идентификатору. Этот метод может быть полезен для пользователей, которые знают идентификатор интересующей их группы и хотят получить о ней подробную информацию.
\item \texttt{findDisciplines} позволяет находит предметы по названию. Это может быть полезно для пользователей, которые ищут информацию о конкретном предмете или хотят найти все предметы, соответствующие определенным критериям.
\item \texttt{disciplineById} позволяет получать информацию о предмете по его идентификатору. Это может быть полезно для пользователей, которые знают идентификатор интересующего их предмета и хотят получить о нем подробную информацию.
\item \texttt{findTeachers} позволяет находить преподавателей по заданным параметрам (например, по имени или фамилии). Это может быть полезно для пользователей, которые ищут конкретного преподавателя или хотят найти всех преподавателей, соответствующих определенным критериям.
\item \texttt{teacherById} позволяет получать информацию о преподавателе по его идентификатору. Этот метод может быть полезен для пользователей, которые знают идентификатор интересующего их преподавателя и хотят получить о нем подробную информацию.
\item \texttt{findRooms} позволяет находить аудитории по заданным параметрам (например, по номеру или корпусу). Это может быть полезно для пользователей, которые ищут конкретную аудиторию или хотят найти все аудитории, соответствующие определенным критериям.
\item \texttt{roomById} позволяет получать информацию о конкретной аудитории по ее идентификатору. Это может быть полезно для пользователей, которые знают идентификатор интересующей их аудитории и хотят получить о ней подробную информацию.
\item \texttt{findRegularLessons} позволяет находить расписание занятий по заданным параметрам. Это может быть полезно для пользователей, которые ищут занятия по определенному предмету, учителю или аудитории, или хотят найти все занятия, соответствующие определенным критериям.
\item \texttt{regularLessonById} позволяет получать информацию о конкретном занятии по его идентификатору. Это может быть полезно для пользователей, которые знают идентификатор интересующего их занятия и хотят получить о нем подробную информацию.
\item \texttt{findDayLessons} позволяет получить расписание занятий по заданным фильтрам (группа, аудитория, преподаватель, предмет) на конкретные даты.
\item \texttt{findDayLessonsForNextDays} позволяет получить расписание занятий на ближайшие дни. Пользователи могут выбрать количество дней, на которые они хотят получить расписание занятий, и фильтры (группа, аудитория, преподаватель, предмет).
\item \texttt{calendarByFilter} возвращает URL-адрес ICS-файла с расписанием занятий, отфильтрованным по заданным параметрам. Этот URL-адрес можно интегрировать в любое приложение календаря для импорта расписания занятий в календарь.
\end{enumerate}

Вся техническая документация сервиса GraphQL API, доступна по ссылке:
\url{https://mtuci.stereobreeze.ru/playground}.
Здесь располагается расширенное описание сервиса, описания всех публичных методов и их параметров,
а также примеры запросов и ответов.

На данный момент сервис активен, и доступен по публичному адресу \url{https://mtuci.stereobreeze.ru/playground}.
Исходный код сервиса доступен по ссылке: \url{https://github.com/alexsvdk/mtuci_rasp_back/tree/master/backend}.