\newpage
\begin{center}
    \textbf{\large АННОТАЦИЯ}
\end{center}


Основная цель проекта заключается в разработке автоматизированной информационной системы расписания для МТУСИ.
Данный комплекс программ предоставляет возможность пользователям просматривать расписание занятий по различным фильтрам и добавлять его в личный календарь.

Для достижения цели работы были поставлены следующие задачи: разработка GraphQl API для доступа к расписанию,
разработка парсера расписания из таблиц exel в электронный формат, разработка мобильного приложения Android и iOS, и 
разработка ICS API для интеграции с приложениями календарей.

В процессе исследования была изучена документация ПО.
В результате были разработаны мобильные приложения с расписаниями МТУСИ в для iOS и Android,
открытое GraphQL API для доступа к расписанию любых программ, и ICS API для интеграции с приложениями.

Теперь участники учебного процесса МТУСИ теперь могут скачать мобильные приложения, и узнавать расписание занятий, 
а желающие внедрить своп ПО могут использовать GraphQL API для получения актуального расписания.
ICS API позволяет пользователям быстро и удобно получать доступ к расписанию занятий университета.

Таким образом, данная работа демонстрирует возможности автоматизации информационных процессов в университете и
значительно упрощает процесс поиска расписания на любой день для любого участника учебного процесса.

\onehalfspacing
\setcounter{page}{2}

\newpage
\renewcommand{\contentsname}{\centerline{\large СОДЕРЖАНИЕ}}
\tableofcontents

\newpage
\begin{center}
\textbf{\large ВВЕДЕНИЕ}
\end{center}
\addcontentsline{toc}{chapter}{ВВЕДЕНИЕ}


\textbf{Цель проекта} -- Разработать комплекс программ для удобного использования расписания занятий университета МТУСИ.
Комплекс программ позволит многократно упростить процесс поиска расписания на любой день для любого участника учебного процесса.
Расписание можно будет искать по учебной группе, преподавателю, учебной дисциплине, кафедре, дню и времени,
или по любому сочетанию этих фильтров.
Доступ к расписанию будет осуществляться через мобильные приложения iOS и android, и интеграцию в персональный календарь.

\textbf{Актуальность проекта} обусловлена внедрением новых технологий для упрощения и оптимизации образовательного процесса. 
Проект позволит значительно сократить время, затрачиваемое участниками учебного процесса (преподавателей, студентом, и других лиц) 
на ознакомление с расписанием занятий. Что повысит эффективность учебного процесса в целом. 
С другой стороны актуальность обусловлена глобальным трендом массового использования мобильных устройств с целью получения информации.
\textbf{Научная новизна проекта} заключается в применении новейших технологий, 
методологий и подходов к разработке программного обеспечения. 
И исследовании их совместимости и сочетаемости между собой. 
Данный проект позволит интегрировать текущие наработки по ведению учебного процесса 
с современными способами получения информации.

В \textbf{задачи проекта} входило:
\begin{enumerate}
    \item Разработка GraphQl API для доступа к расписанию.
    \item Разработка парсера расписания из таблиц exel в электронный формат.
    \item Разработка мобильных приложений Android и iOS.
    \item Разработка ICS API для интеграции с приложениями календарей.
\end{enumerate}
Кроме того, в процессе разработки были дополнительно выполнены следующие задачи:
\begin{enumerate}
    \item Разработка микросервисной архитектуры для распределения нагрузки.
    \item Разработка PWA (Progressive Web App) для удобного использования расписания на компьютере.
\end{enumerate}