\newpage
\thispagestyle{empty}
\begin{center}
  \textbf{\large АННОТАЦИЯ}
\end{center}


Основная цель дипломной работы заключается в разработке автоматизированной информационной системы расписания МТУСИ.
Данный комплекс программ предоставляет возможность пользователям просматривать расписание занятий университета.

Для достижения цели работы были поставлены следующие задачи: 
\begin{enumerate}
    \item Разработка GraphQl API для доступа к расписанию.
    \item Разработка парсера расписания из таблиц exel в электронный формат.
    \item Разработка мобильного приложения Android и iOS.
    \item Разработка ICS API для интеграции с приложениями календарей и разработка чат-бота в ВК и Телеграм.
\end{enumerate}

В процессе исследования была изучена документация ПО.
В результате были разработаны мобильные приложения с расписаниями МТУСИ для Android и iOS, и
открытое GraphQL API для доступа к расписанию любых программ.

Все участники учебного процесса МТУСИ теперь могут пользоваться мобильными приложениями, 
а желающие разработать собственный сервис могут использовать GraphQL API для получения актуального расписания.

Таким образом, данная работа демонстрирует возможности автоматизации информационных процессов в университете и
значительно упрощает процесс поиска расписания на любой день для любого участника учебного процесса.

\onehalfspacing
\setcounter{page}{0}

\newpage
\renewcommand{\contentsname}{\centerline{\large СОДЕРЖАНИЕ}}
\tableofcontents

\newpage
\begin{center}
  \textbf{\large ВВЕДЕНИЕ}
\end{center}
\addcontentsline{toc}{chapter}{ВВЕДЕНИЕ}


\textbf{Актуальность}
Актуальность выпускной квалификационной работы по разработке автоматизированной информационной системы расписания МТУСИ проявляется в нескольких аспектах.
\begin{enumerate}
    \item Проект предлагает использование новых технологий для оптимизации образовательного процесса, что является важным шагом в развитии современного образования. 
    Разработанный комплекс программ значительно сократит время, затрачиваемое на поиск расписания занятий, и обеспечит участникам учебного процесса 
    удобный и быстрый доступ к расписанию.
    \item Актуальность проекта так же обюсуловлена глобальным трендом активного использования мобильных устройств. 
    Разработанные мобильные приложения для iOS и Android позволят пользователям легко получать доступ к расписанию, 
    а также интегрировать его в персональные календари. Это удовлетворит потребности современного общества, где мобильные технологии играют все более важную роль в повседневной жизни.
\end{enumerate}

\textbf{Научной новизна} данного проекта состоит в применении новейших технологий, методологий и подходов к разработке программного обеспечения. 
Разработка GraphQl API, парсера расписания из таблиц Excel, микросервисной архитектуры и Progressive Web App представляют собой инновационные решения, 
которые ранее не применялись в данной области. 
Такое применение новейших технологий позволит интегрировать существующие методы ведения учебного процесса 
с современными способами получения информации, открывая новые возможности для оптимизации образовательных процессов и повышения их эффективности.

\newpage

\textbf{Цель выпускной квалификационной работы} -- Разработать комплекс программ для удобного доступа к расписанию занятий университета МТУСИ.
Комплекс программ позволит многократно упростить и ускорить процесс поиска расписания на любой день для любого участника учебного процесса.
Расписание можно будет искать как по учебной группе, преподавателю, учебной дисциплине, кафедре, дню и времени,
так и по любому сочетанию этих фильтров.
Доступ к расписанию будет осуществляться через мобильные приложения iOS и android, и интеграцию в персональный календарь.

\textbf{Задачи выпускной квалификационной работы:}
\begin{enumerate}
    \item Разработка GraphQl API для доступа к расписанию.
    \item Разработка парсера расписания из таблиц exel в электронный формат.
    \item Разработка мобильных приложений Android и iOS.
    \item Разработка ICS API для интеграции с приложениями календарей.
    \item Разработка микросервисной архитектуры для распределения нагрузки.
    \item Разработка PWA (Progressive Web App) для удобного использования расписания на компьютере.
\end{enumerate}
