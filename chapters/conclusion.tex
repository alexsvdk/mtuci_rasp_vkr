\newpage
\begin{center}
  \textbf{\large ЗАКЛЮЧЕНИЕ}
\end{center}
\refstepcounter{chapter}
\addcontentsline{toc}{chapter}{ЗАКЛЮЧЕНИЕ}


Основные результаты выпускной квалификационной работы:
\begin{enumerate}

\item Разработана информационная система (АИС) для удобного доступа к расписанию МТУСИ. 
АИС включает в себя удобное мобильное приложение, парсер расписания для сбора и обновления данных, 
а также GraphQL API для предоставления доступа к данным расписания.

\item Проведен анализ требований к АИС МТУСИ. Расписание, 
выявлены основные потребности пользователей и определены функциональные возможности, 
которые должна предоставлять система. Также был проведен анализ требований к API для доступа 
к данным расписания и разработан соответствующий интерфейс.

\item Выбраны подходящие технологии для разработки каждого компонента АИС. 
Для сервиса API был выбран язык программирования Kotlin и фреймворк Spring, 
что обеспечивает высокую производительность и гибкость разработки. 
Для парсера расписания были использованы технологии Apache POI и JSOUP, 
позволяющие эффективно собирать и обрабатывать данные расписания. 
Для ICS API выбраны фреймворк Ktor и сервис Amazon S3 для кеширования данных.

\item Разработано удобное мобильное приложение с использованием Flutter, 
которое позволяет пользователям просматривать расписание, искать группы, предметы, 
преподавателей и аудитории, а также импортировать расписание в календарь с помощью ICS-файла.

\item Проведено развертывание АИС на виртуальной машине в Yandex Cloud, 
что обеспечило доступность системы в облаке. Была использована база данных MongoDB 
для хранения данных расписания.

\end{enumerate}
Основные результаты работы позволяют удобно и эффективно получать доступ к 
расписанию МТУСИ через мобильное приложение и API, обеспечивая удобство 
использования и актуальность данных. Разработанная система может быть полезной для 
студентов и сотрудников университета, упрощая процесс планирования и организации учебного процесса.
