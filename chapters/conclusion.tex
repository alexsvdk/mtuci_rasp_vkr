\newpage
\begin{center}
  \textbf{\large ЗАКЛЮЧЕНИЕ}
\end{center}
\refstepcounter{chapter}
\addcontentsline{toc}{chapter}{ЗАКЛЮЧЕНИЕ}


В результате разработки были созданы следующие сервисы, которые выполняют различные функции для обеспечения функциональности приложения.
\begin{enumerate}

    \item GraphQL API. Он предоставляет данные о расписании занятий и экзаменов для приложения
    и использует язык запросов GraphQL для передачи данных между клиентом и сервером.
    Этот сервис обеспечивает доступность данных о расписании занятий для конкретной группы,
    преподавателя, аудитории или предмета на любую дату. Пользователи могут легко получать информацию
    о расписании на нужную дату, что позволяет им планировать свое время и не пропускать важные занятия и экзамены.

    \item Микросервис для генерации ICS-файлов. Он генерирует ICS-файл на основе полученных данных
    о расписании занятий и экзаменов и сохраняет его в файловое распределенное хранилище S3.
    Это позволяет пользователям подписаться на календарь по URL и получать обновленное расписание
    занятий и экзаменов в своем календаре без необходимости ручного импорта файла.
    Благодаря этому сервису, пользователи могут легко получать актуальную информацию
    о своих занятиях и экзаменах, а также установить напоминания для них, чтобы не пропустить важные события.

    \item Сервис Парсинга расписания. Автоматический инструмент для извлечения информации 
    о расписании занятий из электронных таблиц в формате Excel, которые поступают на почтовый ящик.
    С помощью библиотеки Apache POI и других технологий, сервис анализирует файлы,
    извлекает нужную информацию и сохраняет ее в базу данных,
    откуда ее будет использовать сервис GraphQL API.
    Процесс автоматического сбора и анализа информации из различных
    источников данных позволяет получать большие объемы данных и
    использовать их для различных целей, таких как анализ, обработка или хранение.
    Сервис обеспечивает быстрый и эффективный способ работы с информацией
    о расписании занятий, что позволяет улучшить качество образовательного процесса и
    повысить удобство использования информации о расписании занятий для студентов и преподавателей.

    \item Мобильное приложение.Инструмент для получения расписания занятий и экзаменов в МТУСИ.
    Оно содержит ряд ключевых элементов, таких как экран расписания, экран настроек и экран одного занятия,
    которые помогают пользователям быстро находить нужную информацию и
    планировать свое время соответственно. Кроме того, приложение разрабатывается с
    учетом кроссплатформенности и использует современные технологии разработки,
    такие как Flutter и GraphQL API.
    Это позволяет обеспечить высокую доступность и производительность приложения на разных платформах.
    Приложение также использует механизм кеширования данных на стороне клиента,
    что позволяет ускорить загрузку данных и уменьшить нагрузку на сеть и базу данных.
    В целом, описанное приложение является эффективным и удобным инструментом для получения расписания занятий и экзаменов в МТУСИ.

\end{enumerate}
Каждый из этих сервисов имеет важное значение для работы приложения и позволяет
пользователям получать актуальные данные о расписании занятий и экзаменов в удобном формате. 
Благодаря использованию современных технологий разработки, таких как Flutter и GraphQL API,
приложение может быть запущено на различных платформах,
включая iOS, Android и веб-браузеры.
Такой подход позволяет достичь максимальной доступности приложения для пользователей и
обеспечить единый интерфейс и функциональность на разных платформах.

В целом, проект предоставляет пользователям удобный,
интуитивно понятный и многофункциональный интерфейс, который помогает им
оставаться в курсе своих занятий и экзаменов,
легко планировать свое расписание и не пропускать важные события.