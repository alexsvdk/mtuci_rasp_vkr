\newpage
\begin{center}
  \textbf{\large 3. РАЗРАБОТКА ПРОГРАМНОГО ПРОДУКТА}
\end{center}
\refstepcounter{chapter}
\addcontentsline{toc}{chapter}{3. РАЗРАБОТКА ПРОГРАМНОГО ПРОДУКТА}

\section{Выбор инструментов разработки}
Первый этап процесса разработки заключался в выборе инструментов, которые будут использоваться в процессе разработки.

\textbf{IDE IntelliJ Idea} была выбрана в качестве рабочей среды для разработки сервисов.
Она предоставляет широкий набор функций и инструментов для разработки на языке Kotlin.

В качестве тестовой базы данных было решено использовать Docker-образ MongoDB. 
Docker-контейнер позволяет легко развернуть и настроить локальное окружение MongoDB, обеспечивая изолированную и портативную среду разработки. 

\textbf{Visual Studio Code} была выбрана в качестве рабочей среды для разработки мобильного приложения.
Она предоставляет широкий набор функций и инструментов для разработки на языке Dart.

\section{Процесс разработки сервисов}
\subsection{Общий модуль (core)}
Уже перед началом разработки было понятно, что сервисы будут иметь много схожей логики (модели данных, работа с БД, валидация данных, обработка ошибок, логирование и т.д.).
Поэтому для соблюдения принципа DRY 
\footnote{Don’t repeat yourself (DRY; с англ. — «не повторяйся») — это принцип разработки программного обеспечения, нацеленный на снижение повторения информации различного рода, особенно в системах со множеством слоёв абстрагирования.} 
было решено выделить общую часть в отдельный модуль, который будет подключаться к каждому сервису как зависимость. Данный модуль был назван \textbf{core}.
Структурная схема модулей и микросервисов представлена на рисунке \ref{fig:schemes:services}.

\begin{figure}
  \centering
  \includegraphics[width=0.8\linewidth]{images/schemes/services.png}
  \caption{Схема модулей серверной части. Линии показывают направление зависимостей.}
  \label{fig:schemes:services}
\end{figure}

Core модуль инкапсулирует в себе несколько функций: работа с данными, конфигурация, DI
\footnote{Dependency injection (DI) или внедрение зависимостей представляет механизм, который позволяет сделать компоненты программы слабосвязанными, 
а всю программу в целом более гибкой, более адаптируемой и расширяемой.
}, и предоставление общих внешних зависимостей.

Работа с данными включает себя описание моделей данных и репозиториев
\footnote{Репозиторий - это слой абстракции, инкапсулирующий в себе всё, что относится к способу хранения данных. 
Назначение: Разделение бизнес-логики от деталей реализации слоя доступа к данным.}
для работы с БД. Код моделей доступен по ссылке 4.1.\ref{code:models}. Код репозиториев доступен по ссылке 4.1.\ref{code:repo}.

Конфигурация представляет собой класс, который содержит в себе порт, 
адрес БД, настройки email-сервера, настройки S3 хранилища и т.д.
Конфигурация переиспользуется в каждом сервисе.
Код конфигурации доступен по ссылке 4.1.\ref{code:config}.

DI реализован с помощью библиотеки Koin. Core модуль предоставляет общий DI контейнер, 
с уже зарегистрированными фабриками
\footnote{Фабрика – это порождающий паттерн проектирования, который позволяет создавать обьекты, не привязываясь к конкретным реализациям.}
и одиночками 
\footnote{Одиночка (Singleton, Синглтон) - порождающий паттерн, который гарантирует, что для определенного класса будет создан только один объект.}
для всех общих зависимостей. Далее каждый сервис может получить этот контейнер, 
зарегистрировать в нем свои зависимости, не используя конкретные реализации.
Код DI контейнера доступен по ссылке 4.1.\ref{code:di}.

Общие внешние зависимости представляют собой библиотеки, которые используются в каждом сервисе. 
Core модуль предоставляет следующие зависимости:
\begin{enumerate}
  \item graphql-kotlin-federation -- Библиотека для разметки схемы GraphQL.
  \item kmongo -- Библиотека для работы с MongoDB.
  \item koin -- Библиотека для реализации DI.
  \item logback -- Библиотека для логирования.
\end{enumerate}
Исходный код общих внешних зависимостей доступен по ссылке 4.1.\ref{code:gradle}.

Общие утилиты представляют собой классы, которые содержат вспомогательные функции. В контексте данной АИС это:
\begin{enumerate}
  \item Рассчет расписания на определенный период времени для заданного списка предметов.
  \item Рассчет хеш суммы для набора параметров поиска расписания.
\end{enumerate}
Исходный код общих утилит доступен по ссылке 4.1.\ref{code:utils}.

\subsection{Сервис API (backend)}



\section{Процесс разработки мобильного приложения}

\section{Развертывание АИС}

\section{Результаты разработки}