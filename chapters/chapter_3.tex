
\newpage
\begin{center}
    \textbf{\large 3. СЕРВИС ICS API}
\end{center}
\refstepcounter{chapter}
\addcontentsline{toc}{chapter}{3. СЕРВИС ICS API}

Микросервис ICS API предназначен для генерации ICS-файлов с расписанием занятий 
и экзаменов по заданным фильтрам. Пользователи могут выбрать группу, аудиторию, 
преподавателя, предмет и даты, на которые они хотят получить расписание занятий и экзаменов. 
После выбора фильтров, микросервис генерирует ICS-файл, который можно импортировать в календарь, 
такой как Google Calendar или Microsoft Outlook. Кроме того, микросервис предоставляет возможность 
подписаться на календарь по URL. Это позволяет пользователям получать обновленное расписание занятий 
и экзаменов в своем календаре без необходимости ручного импорта файла.

\section{Обзор технологий, использованных при разработке сервиса ICS}

\textbf{ICS (iCalendar)} - это открытый стандарт, 
который определяет формат обмена календарными данными между различными приложениями и устройствами. 
Формат данных, используемый в iCalendar, позволяет описывать события, задачи, напоминания и другие элементы, 
связанные с управлением временем. Файлы в формате ICS могут быть импортированы в календарные приложения, 
такие как Google Календарь, Microsoft Outlook, Apple iCal и другие, 
что позволяет пользователям синхронизировать свои расписания между различными устройствами и приложениями.

\textbf{Ktor} - это асинхронный веб-фреймворк, написанный на языке Kotlin, 
который позволяет быстро создавать и развертывать веб-приложения.
Он использует корутины для обеспечения эффективной обработки запросов и поддержки высоких нагрузок. 
Фреймворк предоставляет множество инструментов для работы с HTTP, 
включая маршрутизацию запросов, обработку запросов и ответов, работу с шаблонами и многое другое.

\textbf{Amazon S3 (Simple Storage Service)} — это хранилище объектов, предоставляемое Amazon Web Services. 
Сервис предоставляет масштабируемую, доступную, безопасную и простую в использовании инфраструктуру для хранения 
и извлечения любых объемов данных из любого места в Интернете. 
Amazon S3 позволяет хранить и извлекать файлы любого размера, включая изображения, 
видео и аудиофайлы, а также любые другие типы файлов. 
Сервис предоставляет широкий набор функций для управления доступом, шифрования и аудита данных, 
а также инструменты для автоматического масштабирования и управления хранилищем данных.

\section{Процесс создания ICS файла}
\begin{enumerate}
\item Пользователь узнает URL ICS файла календаря через метод GraphQL \texttt{calendarByFilter}.
\item При первоначальном обращении по URL микросервис получает фильтры из URL и запрашивает у сервиса 
GraphQL API данные о занятиях и экзаменах, соответствующих заданным фильтрам. 
Затем микросервис генерирует ICS-файл на основе полученных данных и возвращает 
его пользователю и сохраняет в файловае распределенное хранилище S3.
\item При последующих обращениях по URL микросервис проверяет актуальность уже 
генерированного файла при помощи механизма ведения ревизий, и если файл все 
еще актуальный - возвращает HTTP код переадресации на ICS из файлового хранилища.
\end{enumerate}

\section{Результат разработки сервиса ICS API}
На данный момент сервис активен, и его можно использовать по ссылке \url{http://mtuci.stereobreeze.ru/ics/*.ics}, 
где \texttt{*} - это ID фильтра, по которому будет сгенерирован ICS файл.
Исходный код сервиса расположен на доступен по ссылке \url{https://github.com/alexsvdk/mtuci_rasp_back/tree/master/ics_backend}.
