
\textbf{Мобильное приложение} - это программа, которая разработана для работы на мобильных устройствах,
таких как смартфоны и планшеты. Они обычно загружаются из центра приложений на мобильном устройстве,
таком как App Store или Google Play, и устанавливаются на устройство.
Мобильные приложения могут выполнять различные задачи, включая игры, социальные сети,
продуктивные инструменты, торговые платформы, и т.д.
В том числе, мобильные приложения могут быть использованы для доступа участников учебного процесса к расписанию.

Мобильные приложения могут быть созданы для разных операционных систем, 
таких как iOS, Android, Windows Mobile и т.д.
Каждая операционная система имеет свои собственные инструменты и языки программирования для создания мобильных приложений.
Например, для создания мобильных приложений для iOS,
нужно использовать язык программирования Swift или Objective-C, а для Android - Java или Kotlin.

Мобильные приложения имеют множество преимуществ перед другими типами программ.
Они могут использовать различные функции мобильных устройств, такие как камера, GPS, датчики, уведомления и т.д.,
чтобы создать более интуитивный и персонализированный пользовательский опыт.
Также мобильные приложения могут работать в автономном режиме, без подключения к Интернету,
что делает их более удобными для использования в путешествиях или в местах, где связь ограничена.

\textbf{ICS (iCalendar)} - это открытый стандарт, 
который определяет формат обмена календарными данными между различными приложениями и устройствами. 
Формат данных, используемый в iCalendar, позволяет описывать события, задачи, напоминания и другие элементы, 
связанные с управлением временем.

Файлы в формате ICS могут быть импортированы в календарные приложения, 
такие как Google Календарь, Microsoft Outlook, Apple iCal и другие, 
что позволяет пользователям синхронизировать свои расписания между различными устройствами и приложениями.